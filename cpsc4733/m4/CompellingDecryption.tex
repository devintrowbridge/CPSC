\documentclass[format=sigconf]{acmart}
\usepackage[utf8]{inputenc}
\usepackage[english]{babel}

%opening
\title{Compelling Decryption}
\author{Devin Trowbridge}
\affiliation{%
  \department{Computer Science and Software Engineering}
  \institution{Auburn University}
  \email{dkt0003@auburn.edu}
  \city{Huntsville}
  \state{Alabama}
  \country{United States}
}

\begin{abstract}
  In this paper, I argue that forcing individuals or organizations to decypt their data is unethical. My arguments are rooted in deontological ethical theory based primarily on precedent set in US constitutional law and US court cases. I occasionally venture out into precedent set in other parts of history and the world. I chose this deontological framework because the best way to determine what the ethical choice is for new situtations is to examine and draw from the mountain of ethical decisions our society has already made. There is no power higher than the collective, and collectively we have a long lineage of ethical decisions we can make comparisons to help guide us.
\end{abstract}

\begin{document}

\maketitle

\section{Lavabit}

In 2001 Congress signed the Providing Appropriate Tools Required to Intercept and Obstruct Terrorism Act of 2001 into law. Three years later in response to the PATRIOT Act, Ladar Levison launched Lavabit, an encrypted email company designed to protect users' data from prying eyes. In July 2013, the US government attempted to compel Lavabit into providing law enforcement with the cryptographic keys to all of Lavabit's users. Lavabit refused, the court held Lavabit in contempt, and Lavabit suspended operation shortly after. \cite{lavabit} Did the authorities have the right to compel Lavabit to disclose the means of decrypting the email traffic of all Lavabit's customers? Under 5th Amendment privileges, the right to silence, the government did not have the moral authority. The government put Lavabit in the impossible position of destroying its business, perjury, or contempt of court. 

Naive opponents argue that privacy is reserved for those who have something to hide. Such a stance is categorically false. In the US Supreme Court's opinion issued on \textit{McIntyre v. Ohio Elections Commission}, Justice John Paul Stevens writes, ``Anonymity is a shield from the tyranny of the majority. It exemplifies the purpose behind the Bill of Rights, and of the First Amendment in particular: to protect unpopular individuals from retaliation-and their ideas from suppression-at the hand of an intolerant society.'' \cite{stevens} The government, and for that matter no-one, should have the authority to compel individuals or organizations to decrypt their data.

\section{The right to privacy}

Ultimately, you have a right to privacy. You have a right for the government to not spy on you, follow you, unreasonably search your home or belongings, eavesdrop, stalk, wire tap, and so on. A different medium of storage and communication should not have special exceptions to existing rules. The right to privacy is so ingrained in people, according to a search engine for constitutions of the world, 185 out of 203 in force government constitutions mention the ``Right to Privacy.'' \cite{constitue} Even though the right to privacy is so pervasive, it is still a relatively new concept, not showing up in courts or law until very nearly the 20th century. \cite{haydel} However, the unanimous decision by courts, governments, organizations, and people is there is a ``...right to privacy in a 'penumbra' cast by the First, Third, Fourth, Fifth, and Ninth Amendments.'' \cite{haydel} In recent times, the right to privacy has engulfed public debate as the internet makes privacy rights much more apparent. In fact, the European Union passed legislation colloquially know as the ``right to be forgotten'' which was a huge win for privacy advocates. \cite{gdpr} Clearly, privacy is important to everyone from the single individual, up to the largest organizations. Placing exceptions around privacy of encrypted data is inconsistent with values espoused by all.

\section{The right to silence}

Disclosing the means of decryption amounts to testimony. If confessing the key to an encrypted medium would reveal evidence that is incriminating, you should have a right to not disclose the key. Additionally, disclosing a key potentially implies ownership over the contents of the encrypted medium, further incriminating oneself. Stuck between incriminating yourself, perjury, and court holding you in contempt is exactly the situation that the United States' Fifth Amendment is designed to defend against. Avoiding the ``cruel trilemma'' \footnote{A clever moniker for individuals forced into choosing between ``...self-accusation, perjury, or contempt'' \cite{crueltri}} is not unique to to the United states either. In the 17th century, the right to silence appears in a manifesto published in response to the forcible self-incrimination rampant in England at the time. \cite{agreement} Going back hundreds of years, the right to silence is fundamental to our systems of justice. Why would an exception exist for disclosing means of decryption? Why exclude some kinds of testimony, but not others?  

\section{Reasons for Compelled Decryption}

The reason personal freedoms of privacy and silence are under attack is because for the first time in history, there is no way for authorities to access hidden information without the the aid of the key-holder. A more common, yet similar, case is a safe or strongbox. Before the computer, data was stored on paper and kept safe in safes instead of encrypted hard drives. If the owner of safe did not want to testify the combination of the safe to the court, authorities would use other means to access the contents. A locksmith or safe-cracker was sufficient and available to any reasonably funded law enforcement agency. However with modern encryption schemes, breaking into an encrypted device is quite literally impossible. One of the more common modern encryption schemes, RSA-2048 bit, ``...would take a classical computer around 300 trillion years to break.'' \cite{rsa} Understandably, an inability for authorities to have unmatched power frustrates those of a law and order mindset. However, restraining freedoms of silence is at a minimum  antithetical to western culture. Restraining freedoms of privacy is antithetical to all cultures. Concessions for these rights should not exist just because encryption cracking is an inconvenience. John Adams writes about conceding rights to his wife Abigail Adams, “A constitution of government once changed from freedom, can never be restored. Liberty, once lost, is lost forever.” \cite{adams}

\section{Conclusion}

A modern government which holds personal liberty in high regard cannot reasonably expect to force individuals into revealing the means to decrypt encrypted data. To do so is to violate fundamental rights of privacy and silence. Strictly adhering to the dogma laid down by centuries of precedent is challenging in the face of insurmountable obstacles like un-crack-able encryption schemes. However, the alternative of breaking the foundational blocks of our systems of ethics, justice, and law is inconceivable. Exceptions and exclusions to existing precedent does not make logical sense. These challenges against encryption are merely growing pains as the world transitions fully into the information age. Eventually, the law will come to terms with the fact that it does not enjoy the same omniscience and omnipotence it once did before the computer. Throughout history, similar instances of the law lagging behind have sprung up. Similar to the cowboys in the Wild West and pirates in the Americas, eventually the law will catch up to users on the Internet.


\medskip

\bibliographystyle{ACM-Reference-Format}
\bibliography{Module04}

\end{document}
