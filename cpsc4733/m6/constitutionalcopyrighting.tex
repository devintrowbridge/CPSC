\documentclass[format=sigconf]{acmart}
\usepackage[utf8]{inputenc}
\usepackage[english]{babel}

%opening
\title{Constitutional Copyrighting}
\author{Devin Trowbridge}
\affiliation{%
  \department{Computer Science and Software Engineering}
  \institution{Auburn University}
  \email{dkt0003@auburn.edu}
  \city{Huntsville}
  \state{Alabama}
  \country{United States}
}

\begin{document}

\maketitle

On June 19, 2017, US Representative Darrell Issa introduced the CLASSICS Act (H.R. 3301) to the 115th Congress. The bill aimed to provide Federal protection to audio recorded before 1972. \cite{classics} The bill affords so much protection, recordings from 1957 to 1972 will be protected for 110 years, until 1967. \cite{eff} This is seemingly at odds with the Founders' original fourteen year with one fourteen year extension on copyrights. The Founders' aimed to strike a balance between the interests of creators and the interests of the public. Current copyright law which extends copyrights well over a century is grossly at odds with the Constitution's intention of balancing the interests of creators' and the public.

Unfortunately a lopsided power dynamic between creators, e.g. Disney, RIAA, etc., and the public created the disparity that favors creators today. On one side of the argument, giants like Disney for the film industry and RIAA for the music industry want to defend their rights as sole distributers of works artists have created. These organizations want nothing more than to continue to own 100\% of profits from the works of art. This is evidenced most clearly in the title of the CLASSICS bill: \textit{Compensating Legacy Artists for their Songs, Service, and Important Contributions to Society Act}. The Constitution makes no mention of compensating artists. The Constitution does mention promoting the progress of science and useful arts. 

On the other side of the argument, historically, there are no giants. Librarians, historians, and a generally disinterested general public comprise the opposition to multi-billion dollar industries. Since law-making in the United States is determined at all levels by which side has the most money, \cite{lobby} public interests in the copyright law war have been losing for the past century. Librarians simply do not have the financial resources to match the political buying power of industries known for lavish lifestyles.

Modern copyright law has been corrupted by special interests, unobstructed from stomping all over the original intentions of the US Constitution. Fortunately, money on the opposite side of the argument is starting the surface in the 21st Century. Major tech companies like Google are beginning to fight back against out of control copyright laws. \cite{google} With the help of equally powerful special interest groups, hopefully the rest of the 21st century will see a return to reasonable copyright terms that genuinely promote progress.

\medskip

\bibliographystyle{ACM-Reference-Format}
\bibliography{constitutionalcopyrighting}

\end{document}
