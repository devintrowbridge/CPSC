\documentclass[format=sigconf]{acmart}
\usepackage[utf8]{inputenc}
\usepackage[english]{babel}

%opening
\title{Countering Social Media Addiction}
\author{Devin Trowbridge}
\affiliation{%
  \department{Computer Science and Software Engineering}
  \institution{Auburn University}
  \email{dkt0003@auburn.edu}
  \city{Huntsville}
  \state{Alabama}
  \country{United States}}

\begin{document}

\maketitle

\section{Social Media and Gambling}

First and foremost, there is no doubt that social media addiction is a real, verifiable condition \cite{kuss}. Who should be responsible for the ramifications, effects, and fallout of social media addiction? A knee-jerk reaction might be addictions are the responsibility of an individual. However, social media companies (and any other organization that makes an addictive product) study, prototype, test, and engineer ways to make their product more addictive. How can any reasonably average person stand up to weaponized addiction? Social media companies have an ethical obligation to remove mechanism isms designed to target human addiction weaknesses. 

In his 2018 article, Dr. Kruger of University of Michigan, claims, ``Social media copies gambling methods...designed to lock  users into a cycle of addiction'' \cite{kruger}. Similar to social media, though perhaps tangential, are video games. A report by the United Kingdom Parliament concluded video games with loot boxes should be regulated under UK Gambling laws \cite{lootbox} . Another example of tech companies engineering ways to keep you in their application whether its Facebook or Runescape. 

These addictive mechanisms designed to capture attention should be front and center when talking about social media reform. United States Senator Josh Hawly's SMART Act (Social Media Addiction Reduction Technology), he targets addictive features like infinite scrolling, auto-play, and ``streaks'' maintained by interacting with the app. Additionally he proposes apps include a time limit designed to limit daily user interaction \cite{npr}. Mandatory time limits may be a step too far, but removing infinite scrolling and streaks is backed by academic research \cite{kruger}. 

While social media platforms are predictably upset at government attempts at regulation, they are not taking the necessary steps to curb addiction even though they have an ethical obligation to do so. Eventually, and hopefully soon, social media platforms will either decide to do away with gambling-esque mechanisms and features, or face the regulations of governments worldwide.
 
\medskip

\bibliographystyle{ACM-Reference-Format}
\bibliography{Module03}

\end{document}
