\documentclass[11pt, letterpaper, twocolumn, fleqn]{article}
\usepackage[margin=0.5in]{geometry}
\usepackage[utf8]{inputenc}
\usepackage{amsmath}
\usepackage{amssymb}
\usepackage{graphicx}
\graphicspath{ {./images/} }

\let\oldemptyset\emptyset
\let\emptyset\varnothing

\begin{document}
    \renewcommand{\labelenumi}{\alph{enumi}.}
    \renewcommand{\labelenumii}{(\arabic{enumii})}
    
    \paragraph{4.1.1}
        \begin{enumerate}
            \item \includegraphics[scale=.7]{411a}
            \item \includegraphics[scale=.7]{411b}
            \item \includegraphics[scale=.7]{411d}
            \item \includegraphics[scale=.7]{411f}
        \end{enumerate}
    
    \paragraph{4.1.2}    
        \begin{enumerate}
            \item 
                $\begin{bmatrix}
                    1 & 1 & 1 \\
                    0 & 0 & 0 \\
                    0 & 0 & 0
                \end{bmatrix}$
            \addtocounter{enumi}{3}
            \item 
                $\begin{bmatrix}
                    1 & 0 & 0 & 1 \\
                    0 & 1 & 0 & 0 \\
                    0 & 0 & 1 & 0 \\
                    1 & 0 & 0 & 1
                \end{bmatrix}$
            \item 
                $\begin{bmatrix}
                    1 & 0 & 0 & 1 \\
                    1 & 0 & 0 & 0 \\
                    0 & 1 & 1 & 0 \\
                    0 & 0 & 1 & 0 \\
                \end{bmatrix}$
        \end{enumerate}

    \paragraph{4.1.4}
        \begin{enumerate}
            \item \includegraphics[scale=.7]{414a}
            \item \includegraphics[scale=.7]{414b}
            \item \includegraphics[scale=.7]{414c}
            \item \includegraphics[scale=.7]{414d}
        \end{enumerate}
        
    \paragraph{4.2.1}
        \begin{enumerate}
            \item Anti-reflexive because $x < x$ is never true. 
                  Anti-symmetric because if $x < y$ then y cannot be less than x.
                  Transitive because if $x < y$ and $y < z$ then $x < z$
            \item Reflexive because $x \leq x$ is always true.
                  Anti-symmetric because if $x \neq y$ and $x \leq y$ then y cannot be less than x.
                  Transitive because if $x \leq y$ and $y \leq z$ then $x \leq z$
            \item Reflexive because $x^n = x$ when $n=1$.
                  The relation is anti-symmetric because the $x^n = y$ and $y^m = x$ are only true for some positive integer m and n when $x = y$ and $m=n=1$.
                  Transitive because if $x^n=y$ and $y^m=z$ for some positive integer n and m, then $z = (x^n)^m = x^{nm}$. 
            \item Reflexive because for any positive integer x, $x=x*1$ so xDx.
                  Anti-symmetric, x evenly divides y implies $x \leq y$. $y \leq x$ can only be true if $y=x$, therfore, the relation is anti-symmetric.
                  Transitive. xDy implies $y=xn$ for some positive integer n. yDz implies $z=ym$ for some positive integer m. Combining these gives $z=nmx$ which implies that xDz, therfore the relation is transitive.
            \item Reflexive. $|x-x| \leq 2$ becomes $0 \leq 2$ which is always true.
                  Symmetric because $|x-y| = |y-x|$
                  Not transitive because $|0-2| \leq 2$ and $|2-4| \leq 2$ but $|0-4| \nleq 2$
            \item Reflexive because $x-x=0$.
                  Symmetric because $x - y$ is rational and $y - x$ is rational.
                  Transitive because if $x-y$ is rational and $y-z$ rational then $x-z$ is rational.
        \end{enumerate}
    
    \paragraph{4.2.3}
        \begin{enumerate}
            \item No, a reflexive relation implies that for every $x \in R$, it is true that xRx. A anti-reflexive relation implies that for every $x \in R$, it is false that xRx.
            \item No a symmetric relation implies that for every $x \in R$ it is true that xRy and yRx. An anti-symmetric relation implies that for every $x \in R$ it is false that xRy and yRx.
        \end{enumerate}
    
    \paragraph{4.2.5}
        \begin{enumerate}
            \item 
                \begin{enumerate}
                    \item Anti-reflexive because $|A-A| = |\emptyset| = 0 \neq 1$ 
                    \item Neither symmetric nor anti-symmetric. Not symmetric because if $A=\{a,b\}$ and $B=\{b\}$ then $|A-B| = |\{a\}| = 1$ which does not equal $|B-A| = |\emptyset| = 0$. Not anti-symmetric because if $A=\{a,b\}$ and $B=\{b,c\}$ then $A \neq B$ and 
                        $$|A-B| = |\{a\}| = 1 = |\{b\}| = |B-A|$$
                    \item Not transitive. If A = \{a,c\}, B=\{a,b\}, and C=\{b\} then 
                    \begin{align*}
                        |A-B| &= |\{c\}| = 1 \\
                        |B-C| &= |\{a\}| = 1 \\
                        |A-C| &= |\{a,e\}| = 2
                    \end{align*}
                \end{enumerate}
            \item
                \begin{enumerate}
                    \item Anti-reflexive because $A \cap A = A \neq \emptyset$
                    \item Symmetric because $A \cap B = \emptyset = B \cap A$
                    \item Not transitive because if $A = \{a,b\}, B=\{c\}, C=\{a\}$ then $A \cap B = \emptyset$ and $B \cap C = \emptyset$ and $A \cap C = \{a\} \neq \emptyset$
                \end{enumerate}
        \end{enumerate}
    
    \paragraph{4.3.1}
        \begin{enumerate}
            \item 2
            \item 3
            \item c
            \item g
            \item There are no self loops in the graph.
            \item It is not a walk, trail, or path because there is no edge $(f,c)$.
        \end{enumerate}
    
    \paragraph{4.3.2}
        \begin{enumerate}
            \item It is a circuit, but not a cycle.
            \item $\langle d,b,c,g,f,d \rangle$ with a length of 5.
            \item $\langle c,g,f,e,c \rangle$
            \item $\langle d,b,c,g,f,d \rangle$
            \item $\langle d,b,c,f \rangle$
        \end{enumerate}
    
    \paragraph{4.5.1}
        \begin{enumerate}
            \item No
            \item Yes $\langle b,c,f,e \rangle$
            \item No
            \addtocounter{enumi}{1}
            \item Yes $\langle b,c,d,b \rangle$
        \end{enumerate}
    
    \paragraph{4.5.2}
        \begin{enumerate}
            \item \includegraphics[scale=.7]{452a1} \newline
                  \includegraphics[scale=.7]{452a2} \newline
                  \includegraphics[scale=.7]{452a3} \newline
                  \includegraphics[scale=.7]{452a4} 
        \end{enumerate}
    
    \paragraph{4.5.3}
        \begin{enumerate}
            \item \includegraphics[scale=.7]{453a}
            \item \includegraphics[scale=.7]{453b}
        \end{enumerate}
    
    \paragraph{4.6.1}
        \begin{enumerate}
            \item 
                \begin{align*}
                    A &=   \begin{bmatrix}
                            0 & 1 & 0 & 0\\
                            0 & 0 & 1 & 0\\
                            0 & 0 & 0 & 1\\
                            0 & 1 & 0 & 0
                           \end{bmatrix} \\                  
                    A^2 &= \begin{bmatrix}
                            0 & 0 & 1 & 0\\
                            0 & 0 & 0 & 1\\
                            0 & 1 & 0 & 0\\
                            0 & 0 & 1 & 0 
                           \end{bmatrix} \\
                    A^3 &= \begin{bmatrix}
                            0 & 0 & 0 & 1\\
                            0 & 1 & 0 & 0\\
                            0 & 0 & 1 & 0\\
                            0 & 0 & 0 & 1
                           \end{bmatrix} \\
                    A^4 &= \begin{bmatrix}
                            0 & 1 & 0 & 0\\
                            0 & 0 & 1 & 0\\
                            0 & 0 & 0 & 1\\
                            0 & 1 & 0 & 0
                           \end{bmatrix} \\
                    A^+ &= \begin{bmatrix}
                            0 & 1 & 1 & 1\\
                            0 & 1 & 1 & 1\\
                            0 & 1 & 1 & 1\\
                            0 & 1 & 1 & 1
                           \end{bmatrix}
                \end{align*}
        \end{enumerate}
    
    \paragraph{4.6.2}
        \begin{enumerate}
            \item 
                $$A =   \begin{bmatrix}
                        0 & 0 & 0 & 0 & 0 & 1\\
                        1 & 0 & 0 & 0 & 0 & 0\\
                        0 & 1 & 0 & 1 & 0 & 0\\
                        0 & 0 & 1 & 0 & 1 & 0\\
                        0 & 0 & 0 & 0 & 1 & 1\\
                        0 & 1 & 0 & 0 & 0 & 0
                        \end{bmatrix}$$
            \item
                $$A^2 = \begin{bmatrix}
                        0 & 1 & 0 & 0 & 0 & 0\\
                        0 & 0 & 0 & 0 & 0 & 1\\
                        1 & 0 & 1 & 0 & 1 & 0\\
                        0 & 1 & 0 & 1 & 1 & 1\\
                        0 & 1 & 0 & 0 & 1 & 1\\
                        1 & 0 & 0 & 0 & 0 & 0
                        \end{bmatrix}$$
            \item 2, 4, 5, and 6
        \end{enumerate}
    
    \paragraph{4.7.1}
        \begin{enumerate}
            \item J, I, B, A, F
            \item E, D
            \item (A,D),(G,F),(D,B),(H,I)
        \end{enumerate}
    
    \paragraph{4.8.1}
        \begin{enumerate}
            \item Not an equivalence relation because it is not transitive. If x shares only a bio-mother with y and y shares only a bio-father with z, then x and y share no bio parent.
            \item Equivalence Relation. Reflexive because a person x has the same mother as themself. Symmetric because if a person x shares a mother with person y, then y shares a mother with x. Transitive because if y shares a mother with z then x shares the same parent.
            \newline The partition is made up of two subsets, one who share the same mother and the other who do not share that mother.
        \end{enumerate}
    
    \paragraph{4.8.2}
        \begin{enumerate}
            \item 
                \begin{align*}
                 [0] &= \{44,56,4\} \\
                 [1] &= \{13,17\} \\
                 [2] &= \{2,34\} \\
                 [3] &= \{7,99,31\}
                \end{align*}
        \end{enumerate}
    
    \paragraph{4.8.5}
        \begin{enumerate}
            \item Equivalence relation \newline
                Reflexive because $3m=x-x$ becomes $3m=0$ where $m=0$. \newline
                Symmetric because $y-x = -(x-y)$ and if $\frac{x-y}{3}$ is an integer then $\frac{-(x-y)}{3}$ is also an integer. \newline
                Transitive because if
                \begin{align*}
                 x-y &=3m \Rightarrow y=x-3m \\
                 y-z &= 3n \\
                 x-3m-z &= 3n \\
                 x-z &= 3n+3m  \\
                 x-z &= 3(n+m)
                \end{align*}
                Because n and m are both integers they can be represented by some integer p such that $p=n+m$. Because $x-z=3p$, the relation is transitive.
            \item Not an equivalence relation because it is not transitive.  $x=1, y=2, z=4$ yields the following:
                \begin{align*}
                 x+y&=1+2=3 \Rightarrow m=1 \\
                 y+z&=2+4=6 \Rightarrow m=2 \\
                 x+z&=1+4=5 
                \end{align*}
                There is no integer m such that $3m=5$, therfore $x+y=3m$ is not an equivalence relation.
        \end{enumerate}
    
    \paragraph{4.9.1}
        \begin{enumerate}
            \item $1,2,2,2,3,3, 3, 3, 3, 4$ Non-decreasing
            \item $1,1,2,3,5,8,13,21,34,55$ Non-decreasing
            \item $1,2,2,2,2,3, 3, 3, 3, 3$ Non-decreasing
            \item $\frac{1}{1},
                   \frac{1}{2},
                   \frac{1}{3},
                   \frac{1}{4},
                   \frac{1}{5},
                   \frac{1}{6},
                   \frac{1}{7},
                   \frac{1}{8},
                   \frac{1}{9},
                   \frac{1}{10}$ Decreasing, non-increasing
            \item $3,3,3,3,3,3,3,3,3,3$ Non-increasing, non-decreasing
            \item $1,4,9,16,25,36,49,64,81,100$ Increasing, non-decreasing
        \end{enumerate}
    
    \paragraph{4.9.2}
        \begin{enumerate}
            \item Increasing, non-decreasing
            \item Non-increasing
            \item The first three terms $-3,-4,-3$ show that the sequence does not have any of the four listed properties.
        \end{enumerate}
    
    \paragraph{4.9.3}
        \begin{enumerate}
            \item $2,6,18,54,162,486$
            \item $2,5,8,11,14,17$
            \item $27,9,3,1,\frac{1}{3},\frac{1}{9}$
        \end{enumerate}
    
    \paragraph{4.10.1}
        \begin{enumerate}
            \item $1,2,2,4,8,32$
            \item $1,5,13,41,121,365$
            \item $2,1,5,21,110,681$
            \item $4,5,20,100,2000,200000$
        \end{enumerate}
    
    \paragraph{4.11.1}
        \begin{enumerate}
            \item 
        \end{enumerate}
    
    \paragraph{4.11.2}
        \begin{enumerate}
            \item 
        \end{enumerate}
    
    \paragraph{4.11.4}
        \begin{enumerate}
            \item 
        \end{enumerate}
    
    \paragraph{4.12.1}
        \begin{enumerate}
            \item 
        \end{enumerate}
    
    \paragraph{4.12.2}
        \begin{enumerate}
            \item 
        \end{enumerate}
    
    \paragraph{4.12.3}
        \begin{enumerate}
            \item 
        \end{enumerate}
    
    \paragraph{4.13.1}
        \begin{enumerate}
            \item 
        \end{enumerate}
    
    \paragraph{4.13.3}
        \begin{enumerate}
            \item 
        \end{enumerate}
\end{document}
