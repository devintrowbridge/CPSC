\documentclass[11pt, letterpaper, twocolumn, fleqn]{article}
\usepackage[margin=0.5in]{geometry}
\usepackage[utf8]{inputenc}
\usepackage{amsmath}
\usepackage{amssymb}
\usepackage{graphicx}
\graphicspath{ {./images/} }

\let\oldemptyset\emptyset
\let\emptyset\varnothing

\begin{document}
    \renewcommand{\labelenumi}{\alph{enumi}.}
    
    \paragraph{4.1.1}
        \begin{enumerate}
            \item \includegraphics[scale=.7]{411a}
            \item \includegraphics[scale=.7]{411b}
            \item \includegraphics[scale=.7]{411d}
            \item \includegraphics[scale=.7]{411f}
        \end{enumerate}
    
    \paragraph{4.1.2}    
        \begin{enumerate}
            \item 
                $\begin{bmatrix}
                    1 & 1 & 1 \\
                    0 & 0 & 0 \\
                    0 & 0 & 0
                \end{bmatrix}$
            \addtocounter{enumi}{3}
            \item 
                $\begin{bmatrix}
                    1 & 0 & 0 & 1 \\
                    0 & 1 & 0 & 0 \\
                    0 & 0 & 1 & 0 \\
                    1 & 0 & 0 & 1
                \end{bmatrix}$
            \item 
                $\begin{bmatrix}
                    1 & 0 & 0 & 1 \\
                    1 & 0 & 0 & 0 \\
                    0 & 1 & 1 & 0 \\
                    0 & 0 & 1 & 0 \\
                \end{bmatrix}$
        \end{enumerate}

    \paragraph{4.1.4}
        \begin{enumerate}
            \item \includegraphics[scale=.7]{414a}
            \item \includegraphics[scale=.7]{414b}
            \item \includegraphics[scale=.7]{414c}
            \item \includegraphics[scale=.7]{414d}
        \end{enumerate}
        
    \paragraph{4.2.1}
        \begin{enumerate}
            \item Anti-reflexive because $x < x$ is never true. 
                  Anti-symmetric because if $x < y$ then y cannot be less than x.
                  Transitive because if $x < y$ and $y < z$ then $x < z$
            \item Reflexive because $x \leq x$ is always true.
                  Anti-symmetric because if $x \neq y$ and $x \leq y$ then y cannot be less than x.
                  Transitive because if $x \leq y$ and $y \leq z$ then $x \leq z$
            \item Reflexive because $x^n = x$ when $n=1$.
                  The relation is anti-symmetric because the $x^n = y$ and $y^m = x$ are only true for some positive integer m and n when $x = y$ and $m=n=1$.
                  Transitive because if $x^n=y$ and $y^m=z$ for some positive integer n and m, then $z = (x^n)^m = x^{nm}$. 
            \item Reflexive because for any positive integer x, $x=x*1$ so xDx.
                  Anti-symmetric, x evenly divides y implies $x \leq y$. $y \leq x$ can only be true if $y=x$, therfore, the relation is anti-symmetric.
                  Transitive. xDy implies $y=xn$ for some positive integer n. yDz implies $z=ym$ for some positive integer m. Combining these gives $z=nmx$ which implies that xDz, therfore the relation is transitive.
            \item Reflexive. $|x-x| \leq 2$ becomes $0 \leq 2$ which is always true.
                  Symmetric because $|x-y| = |y-x|$
                  Not transitive because $|0-2| \leq 2$ and $|2-4| \leq 2$ but $|0-4| \nleq 2$
            \item Reflexive because $x-x=0$.
                  Symmetric because $x - y$ is rational and $y - x$ is rational.
                  Transitive because if $x-y$ is rational and $y-z$ rational then $x-z$ is rational.
        \end{enumerate}
    
    \paragraph{4.2.3}
        \begin{enumerate}
            \item No, a reflexive relation implies that for every $x \in R$, it is true that xRx. A anti-reflexive relation implies that for every $x \in R$, it is false that xRx.
            \item No a symmetric relation implies that for every $x \in R$ it is true that xRy and yRx. An anti-symmetric relation implies that for every $x \in R$ it is false that xRy and yRx.
        \end{enumerate}
    
    \paragraph{4.2.5}
        \begin{enumerate}
            \item 
                \begin{enumerate}
                    \item Anti-reflexive because $|A-A| = |\emptyset| = 0 \neq 1$ \newline
                    \item Neither symmetric nor anti-symmetric. Not symmetric because if $A=\{a,b\}$ and $B=\{b\}$ then $|A-B| = |\{a\}| = 1$ which does not equal $|B-A| = |\emptyset| = 0$. Not anti-symmetric because if $A=\{a,b\}$ and $B=\{b,c\}$ then $A \neq B$ and 
                  $$|A-B| = |\{a\}| = 1 = |\{b\}| = |B-A|$$
                    \item Not transitive. If A = \{a,c\}, B=\{a,b\}, and C=\{b\} then 
                    \begin{align*}
                        |A-B| &= |\{c\}| = 1 \\
                        |B-C| &= |\{a\}| = 1 \\
                        |A-C| &= |\{a,e\}| = 2
                    \end{align*}

                \end{enumerate}
            \item
        \end{enumerate}
    
    \paragraph{4.3.1}
        \begin{enumerate}
            \item 
        \end{enumerate}
    
    \paragraph{4.3.2}
        \begin{enumerate}
            \item 
        \end{enumerate}
    
    \paragraph{4.5.1}
        \begin{enumerate}
            \item 
        \end{enumerate}
    
    \paragraph{4.5.2}
        \begin{enumerate}
            \item 
        \end{enumerate}
    
    \paragraph{4.5.3}
        \begin{enumerate}
            \item 
        \end{enumerate}
    
    \paragraph{4.6.1}
        \begin{enumerate}
            \item 
        \end{enumerate}
    
    \paragraph{4.6.2}
        \begin{enumerate}
            \item 
        \end{enumerate}
    
    \paragraph{4.7.1}
        \begin{enumerate}
            \item 
        \end{enumerate}
    
    \paragraph{4.8.1}
        \begin{enumerate}
            \item 
        \end{enumerate}
    
    \paragraph{4.8.2}
        \begin{enumerate}
            \item 
        \end{enumerate}
    
    \paragraph{4.8.5}
        \begin{enumerate}
            \item 
        \end{enumerate}
    
    \paragraph{4.9.1}
        \begin{enumerate}
            \item 
        \end{enumerate}
    
    \paragraph{4.9.2}
        \begin{enumerate}
            \item 
        \end{enumerate}
    
    \paragraph{4.9.3}
        \begin{enumerate}
            \item 
        \end{enumerate}
    
    \paragraph{4.10.1}
        \begin{enumerate}
            \item 
        \end{enumerate}
    
    \paragraph{4.11.1}
        \begin{enumerate}
            \item 
        \end{enumerate}
    
    \paragraph{4.11.2}
        \begin{enumerate}
            \item 
        \end{enumerate}
    
    \paragraph{4.11.4}
        \begin{enumerate}
            \item 
        \end{enumerate}
    
    \paragraph{4.12.1}
        \begin{enumerate}
            \item 
        \end{enumerate}
    
    \paragraph{4.12.2}
        \begin{enumerate}
            \item 
        \end{enumerate}
    
    \paragraph{4.12.3}
        \begin{enumerate}
            \item 
        \end{enumerate}
    
    \paragraph{4.13.1}
        \begin{enumerate}
            \item 
        \end{enumerate}
    
    \paragraph{4.13.3}
        \begin{enumerate}
            \item 
        \end{enumerate}
\end{document}
